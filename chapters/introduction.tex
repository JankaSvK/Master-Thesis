\chapter*{Introduction}
\addcontentsline{toc}{chapter}{Introduction}

In the last decades, we witnessed a massive increase in the amount of digital information owned by individuals. Looking back 20-30 years, people used to record only a few hours of their lives, capturing precious moments. Now, according to available estimates, more than 500 hours of videos is uploaded every minute  to YouTube\footnote{\href{https://www.statista.com/statistics/259477/hours-of-video-uploaded-to-youtube-every-minute/}{Statista -- Hours of video uploaded to YouTube every minute}} only. Furthermore, decreasing prices and increasing availability of the recording electronics (especially smartphones) contribute to the amount of multimedia data created every day. It has also become a trend to share videos from day-to-day lives.

The significant increase in the volume of multimedia data opens several new challenges. One of the most vital is the need for effective search and retrieval of stored data. This problem is not only attractive to researchers, but the initiative also comes from the industry. Companies try to help their customers to organize a vast amount of multimedia data (Google Photos, Facebook, OneDrive, MEGA, and others). Those companies often rely on a broad spectrum of techniques used to store and organize the data internally. Unfortunately, users are usually only provided with the most straightforward techniques to filter the data.

Besides attempting to overcome the challenges of querying large volumes of stored data, we will focus on a scenario in which a user searches for one specific image in a dataset. This task of searching for a previously seen multimedia object is often referred to as visual known-item search (\acrshort{kis}). In this thesis, we investigate several known-item search techniques where users provide a few example images as a collage query or browse through images of faces organized in a grid with respect to their similarity.

Known-item search has become a well-known research area \citep{8352047}. According to recent findings \citep{9037125}, most of the known-item search engines incorporate both querying and interactive search functionality. In order to improve the level of developed KIS systems, researchers organize and participate in annual competitions. These efforts help to increase the interest in user-centered multimedia search. One, for example, is Video Browser Showdown\footnote{\url{https://videobrowsershowdown.org/}}, abbreviated to \acrshort{vbs}. For a comparison, TRECVid \citep{2019trecvidawad} is also an annual competition with the main focus on ranking of scenes based on a textual description.

In this thesis, we investigate a couple of approaches to solve the known-item search task:

\begin{enumerate}
  \item Searching by an image collage query.
  \item Searching by browsing in a set of faces from the dataset.
\end{enumerate}

In the first approach, we focus on searching known images by using only images from other sources and their position on a canvas. Users can create a collage of images that reminds them of the searched scene and then browse through the ranked result list looking for the match.

The second approach is a traversal system over the dataset organized by visual similarity of human faces. This brings two partialy contradicting goals. On one hand, we want to show enough of information to the user so they could navigate in the dataset. On the other hand, we do not want to overwhelm them by the sheer amount of faces in the dataset. We tackle this challenge by organizing the faces into multilevel view supporting navigational queries. We implemented the traversal system and tested its abilities.

% although we found out that this technique is less effective than other retrieval techniques presented in this thesis.

The goal of this thesis is to test and implement both approaches, as mentioned earlier. We aim to create a user-friendly interface for smooth user-system interaction.

For the techniques which search by collages, we provide evaluations on an annotated set of queries. We manually created a set of collages, as our queries, and use them to test the proposed system and fine-tune the hyperparameters. We evaluate the proposed methods based on the rank of the target image. Well-performing methods bring the searched items to the foreground, placing them at the top ranks.

We evaluate the second approach -- search based on face similarities -- with the help of a case study. Firstly, we investigate the space of the feature vectors by evaluating the responses from the survey. Then we present our system for dataset exploration. In the end, we provide a preliminary study, comparing the average time needed to find a target face using our system and using a simple linear search.

\subsubsection*{Thesis structure}

The thesis is divided into four main chapters. After the Introduction, we continue by reviewing Related Work (\autoref{ch:related_work}) and Technical background (\autoref{ch:technical_background}). We then continue by defining the task in \autoref{ch:content_based}.

Following the overview chapters, we propose our solutions in two chapters -- search by collage in \autoref{ch:object_location} and search based on the face similarities in \autoref{ch:face_search}. Chapter \ref{ch:evaluation} includes evaluations of the approaches mentioned in the chapter \ref{ch:object_location}.

The end of the thesis is dedicated to the implementation of the aforementioned approaches. This includes the user's guide -- how to interact with the system, and developer's guide -- how to modify the dataset and run the application with the new data.

\bigskip

In summary, we designed and successfully tested a promising approach based on collage queries. The most promising approach splits images into multiple parts. We also developed a traversal system for previewing a dataset of faces.