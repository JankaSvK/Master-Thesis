\chapter*{Introduction}
\addcontentsline{toc}{chapter}{Introduction}

In the last decades, we witnessed a massive jump in the amount of digital information that every person owns. Looking back 20-30 years ago, people used to record only several hours of their lives, capturing only the most valuable moments of life. Now, according to available estimates, more than 500 hours of multimedia data is uploaded every minute only to YouTube\footnote{\url{https://www.statista.com/statistics/259477/hours-of-video-uploaded-to-youtube-every-minute/}}. Furthermore, decreasing prices and availability of the recording electronics (especially cell phones) are contributing to the amount of multimedia data taken every day. It also became a trend to share videos from day-to-day lives.

This significant increase in the volume of multimedia data places several new challenges. One of them is the need for effective search. This problem is not only attractive to the researchers, but the initiatives also come from industry. Companies try to help their customers organize a vast amount of multimedia data (Google Photos, Facebook, OneDrive, MEGA, and others). The companies often have a broad spectrum of solutions on how to store and organize the data internally. Sadly, users are usually provided only with the most straightforward technique to filter the data.

With the increasing size of the stored data it becomes more difficult to find searched items with a query or filter.
 Furthermore, we are interested in a specific challenging scenario where a user searches for one known image in a database. For example, the user recalls a memory, which corresponds to the searched image. This task of searching for a previously seen multimedia object is often referred to as visual known-item search (KIS). In this thesis, we investigate several known-item search techniques where users provide a few example images as a collage query or browse images of faces organized in a grid with respect to their similarity.

Known-item search has become a well-known research area \cite{8352047}. 
According to recent findings \cite{9037125}, known-item search engines have to support both querying and interactive search functionality.
In order to elevate the level of developed KIS systems, researchers organize and participate in annual competitions. These efforts help to increase the interest in user-centered multimedia search. One, for example, is Video Browser Showdown\footnote{\url{https://videobrowsershowdown.org/}}, or shortly VBS. For a comparison, TRECVid (\cite{2019trecvidawad}) is also an annual competition with the main focus on ranking of scenes based on a textual description.

In this thesis, we test several approaches to solve a known-item search task:

\begin{enumerate}
  \item Searching by an image collage query.
  \item Searching by browsing in a set of faces from the dataset.
\end{enumerate}

In the first approach, we focus on searching known images via only example images and their approximate position in the searched image. User can create a collage of images that reminds them of the searched scene and then browse through the ranked result list looking for the match.

The second approach is an experimental test of the possibility of visual traversing through a dataset of faces. To present a user with a feasible amount of faces in one display, we tackle this challenge by organizing the faces into multilevel views supporting navigational queries.

The goal of this thesis is to create a framework to test both approaches, as mentioned earlier. We aim to create a novice-friendly interface for smooth user-system interaction.

We also provide evaluations for a larger dataset and the approaches tested in this thesis. In the evaluations, we measure a rank in result list of the searched image. Lower the rank of the searched item is, the better the technique worked. For the evaluations, we manually constructed a set of collage queries for randomly selected database images. With these queries, we  tested different hyperparameters of the proposed system.

\subsubsection*{Thesis structure}

The thesis is divided into four main blocks. After this chapter, we continue by reviewing Related Work (\autoref{ch:related_work}) and Preliminaries (\autoref{ch:preliminaries}). Related Work focuses on several existing  KIS systems. Preliminaries contain a summary of the theoretical background needed for later chapters.

The second block contain chapters with the task definition, presented approaches, and evaluations.

The third part of the thesis is dedicated to the demo we created to incorporate both our solutions. This includes the user's guide -- how to interact with the system, and Developer's guide -- how to change the dataset or how to encorporate a new approaches.

\bigskip
To sum up our key contribution, we designed and successfully tested a promising approach based on collage queries. The most promising approach divide images in the database into multiple parts. For more info, we invite you to continue reading.